% This is "sig-alternate.tex" V2.1 April 2013
% This file should be compiled with V2.5 of "sig-alternate.cls" May 2012
%
% This example file demonstrates the use of the 'sig-alternate.cls'
% V2.5 LaTeX2e document class file. It is for those submitting
% articles to ACM Conference Proceedings WHO DO NOT WISH TO
% STRICTLY ADHERE TO THE SIGS (PUBS-BOARD-ENDORSED) STYLE.
% The 'sig-alternate.cls' file will produce a similar-looking,
% albeit, 'tighter' paper resulting in, invariably, fewer pages.
%
% ----------------------------------------------------------------------------------------------------------------
% This .tex file (and associated .cls V2.5) produces:
%       1) The Permission Statement
%       2) The Conference (location) Info information
%       3) The Copyright Line with ACM data
%       4) NO page numbers
%
% as against the acm_proc_article-sp.cls file which
% DOES NOT produce 1) thru' 3) above.
%
% Using 'sig-alternate.cls' you have control, however, from within
% the source .tex file, over both the CopyrightYear
% (defaulted to 200X) and the ACM Copyright Data
% (defaulted to X-XXXXX-XX-X/XX/XX).
% e.g.
% \CopyrightYear{2007} will cause 2007 to appear in the copyright line.
% \crdata{0-12345-67-8/90/12} will cause 0-12345-67-8/90/12 to appear in the copyright line.
%
% ---------------------------------------------------------------------------------------------------------------
% This .tex source is an example which *does* use
% the .bib file (from which the .bbl file % is produced).
% REMEMBER HOWEVER: After having produced the .bbl file,
% and prior to final submission, you *NEED* to 'insert'
% your .bbl file into your source .tex file so as to provide
% ONE 'self-contained' source file.
%
% ================= IF YOU HAVE QUESTIONS =======================
% Questions regarding the SIGS styles, SIGS policies and
% procedures, Conferences etc. should be sent to
% Adrienne Griscti (griscti@acm.org)
%
% Technical questions _only_ to
% Gerald Murray (murray@hq.acm.org)
% ===============================================================
%
% For tracking purposes - this is V2.0 - May 2012

\documentclass{sig-alternate-05-2015}
\usepackage{booktabs}


\begin{document}

% Copyright
\setcopyright{acmcopyright}
%\setcopyright{acmlicensed}
%\setcopyright{rightsretained}
%\setcopyright{usgov}
%\setcopyright{usgovmixed}
%\setcopyright{cagov}
%\setcopyright{cagovmixed}


% DOI
\doi{n/a}

% ISBN
\isbn{n/a}

%Conference
%\conferenceinfo{PLDI '13}{June 16--19, 2013, Seattle, WA, USA}

%\acmPrice{\$15.00}

%
% --- Author Metadata here ---
%\conferenceinfo{WOODSTOCK}{'97 El Paso, Texas USA}
%\CopyrightYear{2015} % Allows default copyright year (20XX) to be over-ridden - IF NEED BE.
%\crdata{0-12345-67-8/90/01}  % Allows default copyright data (0-89791-88-6/97/05) to be over-ridden - IF NEED BE.
% --- End of Author Metadata ---

\title{Analysis of Classification Techniques for Prediction of Tuberculosis Defaulters}
%
% You need the command \numberofauthors to handle the 'placement
% and alignment' of the authors beneath the title.
%
% For aesthetic reasons, we recommend 'three authors at a time'
% i.e. three 'name/affiliation blocks' be placed beneath the title.
%
% NOTE: You are NOT restricted in how many 'rows' of
% "name/affiliations" may appear. We just ask that you restrict
% the number of 'columns' to three.
%
% Because of the available 'opening page real-estate'
% we ask you to refrain from putting more than six authors
% (two rows with three columns) beneath the article title.
% More than six makes the first-page appear very cluttered indeed.
%
% Use the \alignauthor commands to handle the names
% and affiliations for an 'aesthetic maximum' of six authors.
% Add names, affiliations, addresses for
% the seventh etc. author(s) as the argument for the
% \additionalauthors command.
% These 'additional authors' will be output/set for you
% without further effort on your part as the last section in
% the body of your article BEFORE References or any Appendices.

%\numberofauthors{8} %  in this sample file, there are a *total*
% of EIGHT authors. SIX appear on the 'first-page' (for formatting
% reasons) and the remaining two appear in the \additionalauthors section.
%
\author{
% You can go ahead and credit any number of authors here,
% e.g. one 'row of three' or two rows (consisting of one row of three
% and a second row of one, two or three).
%
% The command \alignauthor (no curly braces needed) should
% precede each author name, affiliation/snail-mail address and
% e-mail address. Additionally, tag each line of
% affiliation/address with \affaddr, and tag the
% e-mail address with \email.
%
% 1st. author
\alignauthor
Brian Mc George\\
       \affaddr{University of Cape Town}\\
       \affaddr{Cape Town, South Africa}\\
       \email{mcgbri004@myuct.ac.za}
}

\maketitle
\begin{abstract}
\end{abstract}

%
%  Use this command to print the description
%
\printccsdesc

% We no longer use \terms command
%\terms{Theory}

%\keywords{ACM proceedings; \LaTeX; text tagging}

\section{Introduction}
South Africa has the second highest incidence rate of Tuberculosis (TB) per 100 000 of the population \cite{world2015TB}. Its neighbour, Lesotho, has the highest rate of TB per 100 000 of the population \cite{world2015TB}. South Africa has the sixth highest incidence rate of TB in terms of absolute occurrences \cite{world2015TB}. The consequences of defaulting TB treatment include: increased drug resistance, increased health system costs \cite{Lackey:10356751520150601, muture:6660173120110101}, higher risk of mortality, continued risk of transmitting the disease to others \cite{Lackey:10356751520150601} and increased rate of recurrent disease \cite{Jha:10.1371/journal.pone.0008873}. The spread of TB can be reduced if the individuals who have a high risk of defaulting can be predicted. This will also reduce health system costs.

\begin{enumerate}
	\item Discuss datasets used and possibly expansion of TB issues in those specific counties (Peru and Malawi)
	\item Outline briefly how the datasets are used and that each technique is also benchmarked against the well known Australian and German financial to determine how applicable credit scoring research is to TB default prediction for the two TB datasets.
	\item Link TB classification to credit scoring and outline notable similarities and differences
	\item Summarise overall paper
\end{enumerate}

\section{Related work}
\begin{enumerate}
	\item Summarised version of original literature review except the parts using temporal aspects
	\item Summarise what has been done determining TB predictors (which is a different aim)
	\item Summarise what has been done in the credit scoring financial space as a lot of research has been done in this field.
\end{enumerate}

\section{Method}
\begin{enumerate}
	\item Outline 
	\item Brief overview of each classification technique and possibly give a reason why this technique may work well for our application
\end{enumerate}

\section{Results}

\section{Discussion}

\section{Future Work}

\section{Conclusions}

%
% The following two commands are all you need in the
% initial runs of your .tex file to
% produce the bibliography for the citations in your paper.
\bibliographystyle{abbrv}
\bibliography{sigproc}  % sigproc.bib is the name of the Bibliography in this case
% You must have a proper ".bib" file
%  and remember to run:
% latex bibtex latex latex
% to resolve all references
%
%\balancecolumns
\end{document}
